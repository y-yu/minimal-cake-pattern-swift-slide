\definecolor{links}{HTML}{2A1B81}
\hypersetup{colorlinks,linkcolor=,urlcolor=links}

\usetheme{Boadilla}
\usecolortheme{seahorse}
\usefonttheme{serif}
\beamertemplatenavigationsymbolsempty

\setbeamertemplate{bibliography item}{\insertbiblabel}

\usepackage{luacode}
\usepackage{luatexja}
\usepackage{pgfpages}
\usepackage[osf]{mathpazo}

\begin{luacode*}
  USE_IPAFONT = os.getenv"USE_IPAFONT"
  USE_YUFONT = os.getenv"USE_YUFONT"
  
  if USE_YUFONT == "true" then
    tex.sprint("\\AtBeginDocument{\\usepackage[yu-osx, deluxe, expert]{luatexja-preset}}")
  elseif USE_IPAFONT == "true" then
    tex.sprint("\\AtBeginDocument{\\usepackage[ipaex, deluxe, expert]{luatexja-preset}}")
  else
    tex.sprint("\\AtBeginDocument{\\usepackage[hiragino-pro, deluxe, expert]{luatexja-preset}}")
  end
\end{luacode*}

\usepackage{epigraph}
\usepackage{etoolbox}
\usepackage{tikz}
\usepackage{framed}
\usepackage{libertine}
\usepackage{amsmath}
\usepackage{mathtools}
\usepackage{listings}

\renewcommand{\kanjifamilydefault}{\gtdefault}

%\setbeameroption{show notes on second screen=right}

\setmainfont[Numbers=OldStyle, BoldFont=Palatino Bold]{Palatino}
\setsansfont{CMU Sans Serif}
\setmonofont{CMU Typewriter Text}

\input{vc.tex}

\title[\href{http://qiita.com/yyu/items/193cb9fd782d226f50fc}{Minimal Cake Pattern in Swift}]{%
  \href{http://qiita.com/yyu/items/193cb9fd782d226f50fc}{Minimal Cake Pattern in Swift} \\
  {\normalsize \href{http://kbkz.connpass.com/event/37480}{kbkz.tech \#11}}
}
\author{吉村 優}
\date[September 16, 2016]{%
  September 16, 2016 \\%
  {\footnotesize (Commit ID: \GITAbrHash)}
}
\institute[\href{https://twitter.com/\_yyu\_}{@\_yyu\_}]{%
  \url{https://twitter.com/\_yyu\_}\\
  \url{http://qiita.com/yyu}\\
  \url{https://github.com/y-yu}\\
}

\input{./lib/quotebox.tex}
\input{./lib/footnotemark.tex}
\input{./lib/ballon.tex}
\newfontfamily\listingfont{Menlo}
\definecolor{dkgreen}{rgb}{0,0.6,0}
\definecolor{gray}{rgb}{0.5,0.5,0.5}
\definecolor{mauve}{rgb}{0.58,0,0.82}

\lstdefinelanguage{swift}{
  morekeywords={
    func,if,then,else,for,in,while,do,switch,case,default,where,break,continue,fallthrough,return,
    typealias,struct,class,enum,protocol,var,func,let,get,set,willSet,didSet,inout,init,deinit,extension,
    subscript,prefix,operator,infix,postfix,precedence,associativity,left,right,none,convenience,dynamic,
    final,lazy,mutating,nonmutating,optional,override,required,static,unowned,safe,weak,internal,
    private,public,is,as,self,unsafe,dynamicType,true,false,nil,Type,Protocol,
  },
  morecomment=[l]{//},
  morecomment=[s]{/*}{*/},
  morestring=[b]"
}


\makeatletter
\lst@CCPutMacro\lst@ProcessOther {"2D}{\lst@ttfamily{-{}}{-{}}}
\@empty\z@\@empty
\makeatother

\lstset{
  basicstyle=\listingfont,
  frame=single,
  xleftmargin=2em,
  xrightmargin=1em,
  breaklines=true
}

\lstdefinestyle{swift}{
  basicstyle=\listingfont\scriptsize,
  breakatwhitespace=false,
  language=swift,
  captionpos=b,
  commentstyle=\listingfont\scriptsize\color{dkgreen},
  extendedchars=true,
  xleftmargin=2em,
  xrightmargin=1em,
  keepspaces=true,
  keywordstyle=\listingfont\scriptsize\color{blue},
  emphstyle=\listingfont\scriptsize\color{cyan},
  rulecolor=\listingfont\scriptsize\color{black},
  showspaces=false,
  showstringspaces=false,
  showtabs=false,
  stringstyle=\listingfont\scriptsize\color{mauve},
  tabsize=2
}



\newcommand\ballref[1]{%
\tikz \node[circle, shade,ball color=structure.fg,inner sep=0pt,%
  text width=8pt,font=\tiny,align=center] {\color{white}\ref{#1}};
}

\begin{document}

\frame{\maketitle}

\section{自己紹介}
\begin{frame}
  \frametitle{自己紹介}
  
  \begin{columns}
    \begin{column}{0.35\textwidth}
      \centering
      \begin{figure}
        \includegraphics[width=0.95\textwidth]{img/bird2x.png}
      \end{figure}
    \end{column}
    \begin{column}{0.65\textwidth}
      \begin{itemize}
        \item<2-> Scalaを書く仕事に従事
        \item<3-> 趣味は\LaTeX と暗号技術
        \item<4-> Swiftは最近はじめた初心者 
      \end{itemize}
    \end{column}
  \end{columns}
\end{frame}

\section{Dependency Injectionとは?}

\begin{frame}
  \frametitle{DI(Dependency Injection)とは?}

  \pause

  \begin{exampleblock}{Dependency Injectionとは?}
    \begin{shadequote}[r]{\scriptsize Scalaにおける最適なDependency Injectionの方法を考察する\cite{scala-mcp}}
      \begin{itemize}
        \item Dependencyとは実際にサービスなどで使われるオブジェクトである
        \item InjectionとはDependencyオブジェクトを使うオブジェクトに渡すことである
      \end{itemize}
    \end{shadequote}
  \end{exampleblock}
\end{frame}

\section{DIのメリット}

\begin{frame}
  \frametitle{DIのメリット}

  \pause

  \begin{block}{メリット}
    \begin{shadequote}[r]{\scriptsize Scalaにおける最適なDependency Injectionの方法を考察する\cite{scala-mcp}}
      \begin{itemize}
        \item<3-> コンポーネント同士が疎結合になる
        \item<4-> クライアントの動作がカスタマイズ可能になる
        \item<5-> 依存オブジェクトのモック化によるユニットテストが可能になる
      \end{itemize}
    \end{shadequote}
  \end{block}

  \begin{center}
    \uncover<6->{
      \begin{tikzpicture}
        \calloutquote[width=5cm,position={(0.7,-0.2)},fill=blue!30,rounded corners]{どうやってやるの?}
      \end{tikzpicture}
    }
  \end{center}
\end{frame}

\section{Swiftにおける代表的なDI手法}

\begin{frame}
  \frametitle{Swiftにおける代表的なDI手法}

  \pause

  \begin{exampleblock}{Swiftにおける代表的なDI手法}
    \begin{shadequote}[r]{\scriptsize Dependency Injection in Swift 2.x\cite{di-swift2}}
      \begin{itemize}
        \item<3-> \href{https://github.com/Swinject/Swinject}{Swinject}を用いた動的なDI
        \item<4-> Cake Patternを用いた静的なDI
      \end{itemize}
    \end{shadequote}
  \end{exampleblock}

  \begin{center}
    \uncover<5->{
      \begin{tikzpicture}
        \calloutquote[width=5cm,position={(0.7,-0.2)},fill=green!30,rounded corners]{これら以外にはないの?}
      \end{tikzpicture}
    }

    \uncover<6->{
      \begin{tikzpicture}
        \calloutquote[width=11cm,position={(-0.7,-0.2)},fill=cyan!30,rounded corners]{Cake Patternの仲間``Minimal Cake Pattern''を紹介!}
      \end{tikzpicture}
    }
  \end{center}
\end{frame}

\section{Example}

\begin{frame}
  \frametitle{Example}

  こんな機能を作りたい

  \pause

  \begin{exampleblock}{\texttt{HashPasswordService}}
    パスワードを鍵付きハッシュ関数\footnote[frame]{ソルトを使ったハッシュ関数のこと}でハッシュ化する機能
  \end{exampleblock}

  \pause

  \begin{block}{必要な機能}
    \begin{itemize}
      \item<4-> 設定ファイルを読み込む機能
      \item<5-> パスワードをハッシュ化する機能
    \end{itemize}
  \end{block}

  \begin{center}
    \uncover<6->{
      \begin{tikzpicture}
        \calloutquote[width=8cm,position={(0.7,-0.2)},fill=blue!30,rounded corners]{設定ファイルを読み込むのはなぜ?}
      \end{tikzpicture}
    }
  \end{center}
\end{frame}


\begin{frame}
  \frametitle{Example}

  \begin{block}{考えること}
    \begin{itemize}
      \item<2-> ソルトをハードコードするのは微妙
      \item<3-> ソルトは設定ファイルに保存する
      \item<4-> 一方、テストの時はファイルから読み込みたくない
        \begin{center}
          \uncover<5->{
            \begin{tikzpicture}
              \calloutquote[width=9cm,position={(0.7,-0.2)},fill=red!30,rounded corners]{ファイルIOに失敗したらテストが失敗する!}
            \end{tikzpicture}
          }
        \end{center}
      \item<6-> 本実装はソルトを設定ファイルから読み込んで、
        テストの時はハードコードしたソルトを使う
        \begin{center}
          \uncover<7->{
            \begin{tikzpicture}
              \calloutquote[width=5cm,position={(-0.7,-0.2)},fill=green!30,rounded corners]{二つの実装が必要}
            \end{tikzpicture}
          }
        \end{center}
    \end{itemize}
  \end{block}
\end{frame}

\subsection{設定ファイルを読み込む機能}

\begin{frame}
  \frametitle{設定ファイルを読み込む機能}

  まずは設定ファイルを読み込む部分をつくる
  \begin{enumerate}
    \item<2-> インターフェース\lstinline|ReadConfigService|を作成
      \lstinputlisting[style=swift, firstline=3, lastline=7]{src/MCPExample/ReadConfigService.swift}
    \item<3-> 実装を投入
      \lstinputlisting[style=swift, firstline=9, lastline=17]{src/MCPExample/ReadConfigService.swift}
  \end{enumerate}
\end{frame}

\begin{frame}
  \frametitle{設定ファイルを読み込む機能}

  \begin{enumerate}
    \setcounter{enumi}{3}
    \item<1-> メイン実装を\lstinline|ReadConfigServiceImpl|を作成
      \lstinputlisting[style=swift, firstline=23, lastline=34]{src/MCPExample/ReadConfigService.swift}
  \end{enumerate}

  \begin{center}
    \uncover<2->{
      \begin{tikzpicture}
        \calloutquote[width=11cm,position={(-0.7,-0.2)},fill=blue!30,rounded corners]{設定ファイルのパスを引数で受け取って、ファイルをオープンしてソルトを読み込む}
      \end{tikzpicture}
    }
  \end{center}
\end{frame}

\begin{frame}
  \frametitle{設定ファイルを読み込む機能}

  \begin{enumerate}
    \setcounter{enumi}{4}
    \item<1-> モック実装\lstinline|ReadConfigServiceMockImpl|を作成
      \lstinputlisting[style=swift, firstline=36, lastline=47]{src/MCPExample/ReadConfigService.swift}
  \end{enumerate}

  \begin{center}
    \uncover<2->{
      \begin{tikzpicture}
        \calloutquote[width=11cm,position={(0.7,-0.2)},fill=green!30,rounded corners]{設定ファイルのパスはダミー、ソルトもコンストラクタの引数で与えられた値を必ず返す}
      \end{tikzpicture}
    }

    \uncover<3->{
      \begin{tikzpicture}
        \calloutquote[width=9cm,position={(-0.7,-0.2)},fill=red!30,rounded corners]{つまり、設定ファイルにはアクセスしない!}
      \end{tikzpicture}
    }
  \end{center}
\end{frame}

\begin{frame}
  \frametitle{設定ファイルを読み込む機能}

  \begin{enumerate}
    \setcounter{enumi}{5}
    \item<1-> 依存を示すインターフェース\lstinline|UsesReadConfigService|を作成
      \lstinputlisting[style=swift, firstline=19, lastline=21]{src/MCPExample/ReadConfigService.swift}
  \end{enumerate}

  \begin{center}
    \uncover<2->{
      \begin{tikzpicture}
        \calloutquote[width=4cm,position={(-0.7,-0.2)},fill=blue!30,rounded corners]{あとで使います}
      \end{tikzpicture}
    }
  \end{center}
\end{frame}

\subsection{パスワードをハッシュ化する機能}

\begin{frame}
  \frametitle{パスワードをハッシュ化する機能}

  次に、パスワードをハッシュ化する部分を作成
  \begin{enumerate}
    \item<2-> インターフェースを作成
      \lstinputlisting[style=swift, firstline=3, lastline=5]{src/MCPExample/HashPasswordService.swift}
      \begin{center}
        \uncover<3->{
          \begin{tikzpicture}
            \calloutquote[width=10cm,position={(-0.7,-0.2)},fill=green!30,rounded corners]{\lstinline|ReadConfigService|に依存することを示す}
          \end{tikzpicture}
        }
      \end{center}

    \item<4-> 実装を投入
      \lstinputlisting[style=swift, firstline=7, lastline=13]{src/MCPExample/HashPasswordService.swift}
  \end{enumerate}
\end{frame}

\begin{frame}
  \frametitle{パスワードをハッシュ化する機能}

  \begin{enumerate}
    \item<1-> メイン実装を作成
      \lstinputlisting[style=swift, firstline=19, lastline=25]{src/MCPExample/HashPasswordService.swift}
      \begin{center}
        \uncover<2->{
          \begin{tikzpicture}
            \calloutquote[width=8cm,position={(0.7,-0.2)},fill=green!30,rounded corners]{\lstinline|ReadConfigService|のメイン実装をDI}
          \end{tikzpicture}
        }
      \end{center}

    \item<3-> テスト実装を作成
      \lstinputlisting[style=swift, firstline=27, lastline=30]{src/MCPExample/HashPasswordService.swift}
      \begin{center}
        \uncover<4->{
          \begin{tikzpicture}
            \calloutquote[width=8cm,position={(-0.7,-0.2)},fill=cyan!30,rounded corners]{\lstinline|ReadConfigService|のモック実装をDI}
          \end{tikzpicture}
        }
      \end{center}
  \end{enumerate}
\end{frame}

\begin{frame}
  \frametitle{パスワードをハッシュ化する機能}

  \begin{enumerate}
    \setcounter{enumi}{10}
    \item<1-> モックも作成
      \lstinputlisting[style=swift, firstline=32, lastline=38]{src/MCPExample/HashPasswordService.swift}
      \begin{center}
        \uncover<2->{
          \begin{tikzpicture}
            \calloutquote[width=7cm,position={(0.7,-0.2)},fill=blue!30,rounded corners]{モックはハッシュ値を計算しない}
          \end{tikzpicture}
        }

        \uncover<3->{
          \begin{tikzpicture}
            \calloutquote[width=11cm,position={(-0.7,-0.2)},fill=green!30,rounded corners]{たとえこのサービスの実装が変更されても、依存するサービスのテストは落ちない}
          \end{tikzpicture}
        }
      \end{center}
  \end{enumerate}
\end{frame}

\subsection{テスト}

\begin{frame}
  \frametitle{テスト}

  作成した\texttt{HashPasswordService}のテストを作る
  \uncover<2-> {
    \lstinputlisting[style=swift]{src/MCPExampleTests/HashPasswordServiceTest.swift}
  }
  
  \begin{center}
    \uncover<3->{
      \begin{tikzpicture}
        \calloutquote[width=5cm,position={(0.7,-0.2)},fill=green!30,rounded corners]{設定ファイルを読まない!}
      \end{tikzpicture}
    }
  \end{center}
\end{frame}

\subsection{DIの漏れ}

\begin{frame}[fragile]
  \frametitle{DIの漏れ}

  \pause

  プログラマのミスで次のようにDIを忘れることがある
  \lstinputlisting[style=swift, firstline=40, lastline=44]{src/MCPExample/HashPasswordService.swift}

  \pause

  \begin{alertblock}{コンパイルエラー}
\begin{lstlisting}[basicstyle=\listingfont\scriptsize]
HashPasswordService.swift:41:7: Type 'HashPasswordServiceNgImpl'
does not conform to protocol 'UsesReadConfigService'
\end{lstlisting}
  \end{alertblock}

  \pause

  \begin{center}
    \begin{tikzpicture}
      \calloutquote[width=5cm,position={(-0.7,-0.2)},fill=cyan!30,rounded corners]{実行前にDI漏れを検出!}
    \end{tikzpicture}

    \pause

    \begin{tikzpicture}
      \calloutquote[width=5cm,position={(0.7,-0.2)},fill=green!30,rounded corners]{依存が増えると効果的!}
    \end{tikzpicture}
  \end{center}
\end{frame}

\begin{frame}[fragile]
  \frametitle{DIの漏れ}

  次のように依存が増えていくと、DI漏れがありえてくる
\begin{lstlisting}[style=swift]
protocol CreateUserService: 
  UsesSessionService,
  UsesUserRepository,
  UsesClock,
  UsesApplicationLogger,
  UsesUserConfig,
  UsesRandomGenerator {

  func create(user: User) -> Future<Session>
}
\end{lstlisting}

  \pause

  \begin{center}
    \begin{tikzpicture}
      \calloutquote[width=9cm,position={(0.7,-0.2)},fill=red!30,rounded corners]{大きなアプリケーションを起動させるのは大変!}
    \end{tikzpicture}

    \pause
    
    \begin{tikzpicture}
      \calloutquote[width=10cm,position={(-0.7,-0.2)},fill=green!30,rounded corners]{起動させなくてもDI漏れが検出できるのは便利!}
    \end{tikzpicture}
  \end{center}
\end{frame}

\section{まとめ}

\begin{frame}
  \frametitle{まとめ}
  
  \begin{block}{Minimal Cake Patternのメリット}
    \begin{itemize}
      \item<2-> コンパイラによる静的なチェックにより、DI漏れを検出できる
      \item<3-> DIのための特別なライブラリが必要ない
      \item<4-> 普通のCake Patternに比べてシンプルである
    \end{itemize}
  \end{block}
\end{frame}

\begin{frame}
  \frametitle{目次}

  \tableofcontents
\end{frame}

\section*{参考文献}

\begin{frame}
  \frametitle{参考文献}

  \bibliographystyle{junsrt_url}
  \nocite{*}
  \bibliography{ref}
\end{frame}

\begin{frame}
  \begin{center}
    {\Huge Thank you for listening!}

    \quad \quad

    {\Huge Any question?}
  \end{center}
\end{frame}

\end{document}
